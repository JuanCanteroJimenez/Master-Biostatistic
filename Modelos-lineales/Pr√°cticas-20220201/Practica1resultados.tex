% Options for packages loaded elsewhere
\PassOptionsToPackage{unicode}{hyperref}
\PassOptionsToPackage{hyphens}{url}
%
\documentclass[
]{article}
\usepackage{amsmath,amssymb}
\usepackage{lmodern}
\usepackage{ifxetex,ifluatex}
\ifnum 0\ifxetex 1\fi\ifluatex 1\fi=0 % if pdftex
  \usepackage[T1]{fontenc}
  \usepackage[utf8]{inputenc}
  \usepackage{textcomp} % provide euro and other symbols
\else % if luatex or xetex
  \usepackage{unicode-math}
  \defaultfontfeatures{Scale=MatchLowercase}
  \defaultfontfeatures[\rmfamily]{Ligatures=TeX,Scale=1}
\fi
% Use upquote if available, for straight quotes in verbatim environments
\IfFileExists{upquote.sty}{\usepackage{upquote}}{}
\IfFileExists{microtype.sty}{% use microtype if available
  \usepackage[]{microtype}
  \UseMicrotypeSet[protrusion]{basicmath} % disable protrusion for tt fonts
}{}
\makeatletter
\@ifundefined{KOMAClassName}{% if non-KOMA class
  \IfFileExists{parskip.sty}{%
    \usepackage{parskip}
  }{% else
    \setlength{\parindent}{0pt}
    \setlength{\parskip}{6pt plus 2pt minus 1pt}}
}{% if KOMA class
  \KOMAoptions{parskip=half}}
\makeatother
\usepackage{xcolor}
\IfFileExists{xurl.sty}{\usepackage{xurl}}{} % add URL line breaks if available
\IfFileExists{bookmark.sty}{\usepackage{bookmark}}{\usepackage{hyperref}}
\hypersetup{
  pdftitle={Practica 1},
  pdfauthor={Juan Cantero Jimenez},
  hidelinks,
  pdfcreator={LaTeX via pandoc}}
\urlstyle{same} % disable monospaced font for URLs
\usepackage[margin=1in]{geometry}
\usepackage{color}
\usepackage{fancyvrb}
\newcommand{\VerbBar}{|}
\newcommand{\VERB}{\Verb[commandchars=\\\{\}]}
\DefineVerbatimEnvironment{Highlighting}{Verbatim}{commandchars=\\\{\}}
% Add ',fontsize=\small' for more characters per line
\usepackage{framed}
\definecolor{shadecolor}{RGB}{248,248,248}
\newenvironment{Shaded}{\begin{snugshade}}{\end{snugshade}}
\newcommand{\AlertTok}[1]{\textcolor[rgb]{0.94,0.16,0.16}{#1}}
\newcommand{\AnnotationTok}[1]{\textcolor[rgb]{0.56,0.35,0.01}{\textbf{\textit{#1}}}}
\newcommand{\AttributeTok}[1]{\textcolor[rgb]{0.77,0.63,0.00}{#1}}
\newcommand{\BaseNTok}[1]{\textcolor[rgb]{0.00,0.00,0.81}{#1}}
\newcommand{\BuiltInTok}[1]{#1}
\newcommand{\CharTok}[1]{\textcolor[rgb]{0.31,0.60,0.02}{#1}}
\newcommand{\CommentTok}[1]{\textcolor[rgb]{0.56,0.35,0.01}{\textit{#1}}}
\newcommand{\CommentVarTok}[1]{\textcolor[rgb]{0.56,0.35,0.01}{\textbf{\textit{#1}}}}
\newcommand{\ConstantTok}[1]{\textcolor[rgb]{0.00,0.00,0.00}{#1}}
\newcommand{\ControlFlowTok}[1]{\textcolor[rgb]{0.13,0.29,0.53}{\textbf{#1}}}
\newcommand{\DataTypeTok}[1]{\textcolor[rgb]{0.13,0.29,0.53}{#1}}
\newcommand{\DecValTok}[1]{\textcolor[rgb]{0.00,0.00,0.81}{#1}}
\newcommand{\DocumentationTok}[1]{\textcolor[rgb]{0.56,0.35,0.01}{\textbf{\textit{#1}}}}
\newcommand{\ErrorTok}[1]{\textcolor[rgb]{0.64,0.00,0.00}{\textbf{#1}}}
\newcommand{\ExtensionTok}[1]{#1}
\newcommand{\FloatTok}[1]{\textcolor[rgb]{0.00,0.00,0.81}{#1}}
\newcommand{\FunctionTok}[1]{\textcolor[rgb]{0.00,0.00,0.00}{#1}}
\newcommand{\ImportTok}[1]{#1}
\newcommand{\InformationTok}[1]{\textcolor[rgb]{0.56,0.35,0.01}{\textbf{\textit{#1}}}}
\newcommand{\KeywordTok}[1]{\textcolor[rgb]{0.13,0.29,0.53}{\textbf{#1}}}
\newcommand{\NormalTok}[1]{#1}
\newcommand{\OperatorTok}[1]{\textcolor[rgb]{0.81,0.36,0.00}{\textbf{#1}}}
\newcommand{\OtherTok}[1]{\textcolor[rgb]{0.56,0.35,0.01}{#1}}
\newcommand{\PreprocessorTok}[1]{\textcolor[rgb]{0.56,0.35,0.01}{\textit{#1}}}
\newcommand{\RegionMarkerTok}[1]{#1}
\newcommand{\SpecialCharTok}[1]{\textcolor[rgb]{0.00,0.00,0.00}{#1}}
\newcommand{\SpecialStringTok}[1]{\textcolor[rgb]{0.31,0.60,0.02}{#1}}
\newcommand{\StringTok}[1]{\textcolor[rgb]{0.31,0.60,0.02}{#1}}
\newcommand{\VariableTok}[1]{\textcolor[rgb]{0.00,0.00,0.00}{#1}}
\newcommand{\VerbatimStringTok}[1]{\textcolor[rgb]{0.31,0.60,0.02}{#1}}
\newcommand{\WarningTok}[1]{\textcolor[rgb]{0.56,0.35,0.01}{\textbf{\textit{#1}}}}
\usepackage{graphicx}
\makeatletter
\def\maxwidth{\ifdim\Gin@nat@width>\linewidth\linewidth\else\Gin@nat@width\fi}
\def\maxheight{\ifdim\Gin@nat@height>\textheight\textheight\else\Gin@nat@height\fi}
\makeatother
% Scale images if necessary, so that they will not overflow the page
% margins by default, and it is still possible to overwrite the defaults
% using explicit options in \includegraphics[width, height, ...]{}
\setkeys{Gin}{width=\maxwidth,height=\maxheight,keepaspectratio}
% Set default figure placement to htbp
\makeatletter
\def\fps@figure{htbp}
\makeatother
\setlength{\emergencystretch}{3em} % prevent overfull lines
\providecommand{\tightlist}{%
  \setlength{\itemsep}{0pt}\setlength{\parskip}{0pt}}
\setcounter{secnumdepth}{-\maxdimen} % remove section numbering
\ifluatex
  \usepackage{selnolig}  % disable illegal ligatures
\fi

\title{Practica 1}
\author{Juan Cantero Jimenez}
\date{2/20/2022}

\begin{document}
\maketitle

\hypertarget{la-media-y-la-mediana-son-dos-estimadores-de-tendencia-central-en-distribuciones-ampliamente-conocidos-y-utilizados.-en-esta-tarea-nos-vamos-a-plantear-su-comparaciuxf3n-como-estimadores-de-la-media-de-una-distribuciuxf3n-normal.-para-ello-vamos-a-hacer-uso-de-procedimientos-de-tipo-empuxedrico-muxe1s-que-de-razonamientos-teuxf3ricos.-asuxed}{%
\subsection{1 La media y la mediana son dos estimadores de tendencia
central en distribuciones, ampliamente conocidos y utilizados. En esta
tarea nos vamos a plantear su comparación como estimadores de la media
de una distribución Normal. Para ello vamos a hacer uso de
procedimientos de tipo empírico más que de razonamientos teóricos.
Así:}\label{la-media-y-la-mediana-son-dos-estimadores-de-tendencia-central-en-distribuciones-ampliamente-conocidos-y-utilizados.-en-esta-tarea-nos-vamos-a-plantear-su-comparaciuxf3n-como-estimadores-de-la-media-de-una-distribuciuxf3n-normal.-para-ello-vamos-a-hacer-uso-de-procedimientos-de-tipo-empuxedrico-muxe1s-que-de-razonamientos-teuxf3ricos.-asuxed}}

\begin{Shaded}
\begin{Highlighting}[]
\NormalTok{result\_normal }\OtherTok{\textless{}{-}} \FunctionTok{replicate}\NormalTok{(}\DecValTok{100}\NormalTok{, \{}
\NormalTok{  data }\OtherTok{\textless{}{-}} \FunctionTok{rnorm}\NormalTok{(}\DecValTok{50}\NormalTok{)}
  \FunctionTok{return}\NormalTok{(}\FunctionTok{c}\NormalTok{(}\AttributeTok{media=}\FunctionTok{mean}\NormalTok{(data), }\AttributeTok{mediana =} \FunctionTok{median}\NormalTok{(data)))}
\NormalTok{\})}
\NormalTok{result\_student }\OtherTok{\textless{}{-}} \FunctionTok{replicate}\NormalTok{(}\DecValTok{100}\NormalTok{, \{}
\NormalTok{  data }\OtherTok{\textless{}{-}} \FunctionTok{rt}\NormalTok{(}\DecValTok{50}\NormalTok{,}\DecValTok{1}\NormalTok{)}
  \FunctionTok{return}\NormalTok{(}\FunctionTok{c}\NormalTok{(}\AttributeTok{media=}\FunctionTok{mean}\NormalTok{(data), }\AttributeTok{mediana=}\FunctionTok{median}\NormalTok{(data)))}
\NormalTok{\})}
\FunctionTok{cat}\NormalTok{(}\StringTok{"ECM media muestra normal"}\NormalTok{, }\FunctionTok{mean}\NormalTok{((}\DecValTok{0}\SpecialCharTok{{-}}\NormalTok{result\_normal[}\StringTok{"media"}\NormalTok{,])}\SpecialCharTok{\^{}}\DecValTok{2}\NormalTok{))}
\end{Highlighting}
\end{Shaded}

\begin{verbatim}
## ECM media muestra normal 0.02143283
\end{verbatim}

\begin{Shaded}
\begin{Highlighting}[]
\FunctionTok{cat}\NormalTok{(}\StringTok{"ECM mediana muestra normal"}\NormalTok{, }\FunctionTok{mean}\NormalTok{((}\DecValTok{0}\SpecialCharTok{{-}}\NormalTok{result\_normal[}\StringTok{"mediana"}\NormalTok{,])}\SpecialCharTok{\^{}}\DecValTok{2}\NormalTok{))}
\end{Highlighting}
\end{Shaded}

\begin{verbatim}
## ECM mediana muestra normal 0.03009464
\end{verbatim}

\begin{Shaded}
\begin{Highlighting}[]
\FunctionTok{cat}\NormalTok{(}\StringTok{"ECM media muestra student"}\NormalTok{, }\FunctionTok{mean}\NormalTok{((}\DecValTok{0}\SpecialCharTok{{-}}\NormalTok{result\_student[}\StringTok{"media"}\NormalTok{,])}\SpecialCharTok{\^{}}\DecValTok{2}\NormalTok{))}
\end{Highlighting}
\end{Shaded}

\begin{verbatim}
## ECM media muestra student 164.7917
\end{verbatim}

\begin{Shaded}
\begin{Highlighting}[]
\FunctionTok{cat}\NormalTok{(}\StringTok{"ECM mediana muestra student"}\NormalTok{, }\FunctionTok{mean}\NormalTok{((}\DecValTok{0}\SpecialCharTok{{-}}\NormalTok{result\_student[}\StringTok{"mediana"}\NormalTok{,])}\SpecialCharTok{\^{}}\DecValTok{2}\NormalTok{))}
\end{Highlighting}
\end{Shaded}

\begin{verbatim}
## ECM mediana muestra student 0.05203244
\end{verbatim}

\hypertarget{supongamos-que-disponemos-de-la-siguiente-muestra-de-valores-set.seed1-x---exprnorm50-todos-ellos-valores-positivos-en-la-recta-real.-para-este-conjunto-de-datos-nos-planteamos-ajustarles-una-distribuciuxf3n-gamma-adecuada-para-este-tipo-de-datos-con-valores-positivos.-halla-haciendo-uso-de-r-los-estimadores-mle-de-ux3b1-y-ux3b2-y-representa-un-histograma-de-la-muestra-de-valores-x-con-la-distribuciuxf3n-gamma-que-hayas-estimado-superpuesta.-haciendo-uso-de-la-aproximaciuxf3n-normal-de-los-mle-calcula-un-intervalo-de-confianza-al-95-para-el-paruxe1metro-ux3b1-de-la-distribuciuxf3n-que-acabas-de-calcular.}{%
\subsection{2 Supongamos que disponemos de la siguiente muestra de
valores: set.seed(1); x \textless- exp(rnorm(50)), todos ellos valores
positivos en la recta real. Para este conjunto de datos, nos planteamos
ajustarles una distribución Gamma), adecuada para este tipo de datos con
valores positivos. Halla, haciendo uso de R, los estimadores MLE de α y
β y representa un histograma de la muestra de valores x, con la
distribución Gamma que hayas estimado superpuesta. Haciendo uso de la
aproximación Normal de los MLE calcula un intervalo de confianza al 95\%
para el parámetro α de la distribución que acabas de
calcular.}\label{supongamos-que-disponemos-de-la-siguiente-muestra-de-valores-set.seed1-x---exprnorm50-todos-ellos-valores-positivos-en-la-recta-real.-para-este-conjunto-de-datos-nos-planteamos-ajustarles-una-distribuciuxf3n-gamma-adecuada-para-este-tipo-de-datos-con-valores-positivos.-halla-haciendo-uso-de-r-los-estimadores-mle-de-ux3b1-y-ux3b2-y-representa-un-histograma-de-la-muestra-de-valores-x-con-la-distribuciuxf3n-gamma-que-hayas-estimado-superpuesta.-haciendo-uso-de-la-aproximaciuxf3n-normal-de-los-mle-calcula-un-intervalo-de-confianza-al-95-para-el-paruxe1metro-ux3b1-de-la-distribuciuxf3n-que-acabas-de-calcular.}}

\begin{Shaded}
\begin{Highlighting}[]
\FunctionTok{set.seed}\NormalTok{(}\DecValTok{1}\NormalTok{)}
\NormalTok{x }\OtherTok{\textless{}{-}} \FunctionTok{exp}\NormalTok{(}\FunctionTok{rnorm}\NormalTok{(}\DecValTok{50}\NormalTok{))}
\NormalTok{minusloglikelihod }\OtherTok{\textless{}{-}} \ControlFlowTok{function}\NormalTok{(alpha, beta)\{}\SpecialCharTok{{-}}\FunctionTok{sum}\NormalTok{(}\FunctionTok{dgamma}\NormalTok{(x, alpha, beta, }\AttributeTok{log=}\ConstantTok{TRUE}\NormalTok{))\}}
\NormalTok{fit1}\OtherTok{\textless{}{-}}\NormalTok{stats4}\SpecialCharTok{::}\FunctionTok{mle}\NormalTok{(minusloglikelihod, }\AttributeTok{start=}\FunctionTok{list}\NormalTok{(}\AttributeTok{alpha=}\FloatTok{0.1}\NormalTok{, }\AttributeTok{beta=}\FloatTok{0.1}\NormalTok{))}
\end{Highlighting}
\end{Shaded}

\begin{verbatim}
## Warning in dgamma(x, alpha, beta, log = TRUE): NaNs produced

## Warning in dgamma(x, alpha, beta, log = TRUE): NaNs produced

## Warning in dgamma(x, alpha, beta, log = TRUE): NaNs produced

## Warning in dgamma(x, alpha, beta, log = TRUE): NaNs produced

## Warning in dgamma(x, alpha, beta, log = TRUE): NaNs produced

## Warning in dgamma(x, alpha, beta, log = TRUE): NaNs produced

## Warning in dgamma(x, alpha, beta, log = TRUE): NaNs produced

## Warning in dgamma(x, alpha, beta, log = TRUE): NaNs produced
\end{verbatim}

\begin{Shaded}
\begin{Highlighting}[]
\NormalTok{fit1}\SpecialCharTok{@}\NormalTok{coef}
\end{Highlighting}
\end{Shaded}

\begin{verbatim}
##    alpha     beta 
## 1.896841 1.288558
\end{verbatim}

\begin{Shaded}
\begin{Highlighting}[]
\FunctionTok{hist}\NormalTok{(x, }\AttributeTok{prob =} \ConstantTok{TRUE}\NormalTok{, }\AttributeTok{col=}\StringTok{"gray"}\NormalTok{)}
\FunctionTok{lines}\NormalTok{((}\DecValTok{0}\SpecialCharTok{:}\DecValTok{5000}\NormalTok{)}\SpecialCharTok{/}\DecValTok{1000}\NormalTok{, }\FunctionTok{dgamma}\NormalTok{((}\DecValTok{0}\SpecialCharTok{:}\DecValTok{5000}\NormalTok{)}\SpecialCharTok{/}\DecValTok{1000}\NormalTok{, fit1}\SpecialCharTok{@}\NormalTok{coef[}\DecValTok{1}\NormalTok{], fit1}\SpecialCharTok{@}\NormalTok{coef[}\DecValTok{2}\NormalTok{]), }\AttributeTok{lwd=}\DecValTok{2}\NormalTok{, }\AttributeTok{col=}\StringTok{"red"}\NormalTok{)}
\end{Highlighting}
\end{Shaded}

\includegraphics{Practica1resultados_files/figure-latex/unnamed-chunk-2-1.pdf}

\begin{Shaded}
\begin{Highlighting}[]
\NormalTok{fit1}\SpecialCharTok{@}\NormalTok{coef[}\DecValTok{1}\NormalTok{] }\SpecialCharTok{+} \FunctionTok{c}\NormalTok{(}\SpecialCharTok{{-}}\DecValTok{1}\NormalTok{,}\DecValTok{1}\NormalTok{)}\SpecialCharTok{*}\FunctionTok{qnorm}\NormalTok{(}\FloatTok{0.975}\NormalTok{)}\SpecialCharTok{*}\NormalTok{(}\FunctionTok{sqrt}\NormalTok{(fit1}\SpecialCharTok{@}\NormalTok{vcov[}\DecValTok{1}\NormalTok{,}\DecValTok{1}\NormalTok{])}\SpecialCharTok{/}\FunctionTok{sqrt}\NormalTok{(}\DecValTok{50}\NormalTok{))}
\end{Highlighting}
\end{Shaded}

\begin{verbatim}
## [1] 1.799519 1.994163
\end{verbatim}

\hypertarget{reproduce-por-ti-mismo-el-ejemplo-de-la-puxe1gina-17-del-tema-1-de-la-asignatura.-comprueba-que-los-resultados-que-obtienes-en-cuanto-a-la-proporciuxf3n-de-veces-que-los-intervalos-de-confianza-contienen-el-valor-0-son-similares-a-los-de-los-apuntes}{%
\subsection{3. Reproduce por ti mismo el ejemplo de la página 17 del
Tema 1 de la asignatura. Comprueba que los resultados que obtienes en
cuanto a la proporción de veces que los intervalos de confianza
contienen el valor 0 son similares a los de los
apuntes}\label{reproduce-por-ti-mismo-el-ejemplo-de-la-puxe1gina-17-del-tema-1-de-la-asignatura.-comprueba-que-los-resultados-que-obtienes-en-cuanto-a-la-proporciuxf3n-de-veces-que-los-intervalos-de-confianza-contienen-el-valor-0-son-similares-a-los-de-los-apuntes}}

\begin{Shaded}
\begin{Highlighting}[]
\NormalTok{result }\OtherTok{\textless{}{-}} \FunctionTok{replicate}\NormalTok{(}\DecValTok{1000}\NormalTok{, \{}
\NormalTok{  data }\OtherTok{\textless{}{-}} \FunctionTok{rnorm}\NormalTok{(}\DecValTok{100}\NormalTok{)}
\NormalTok{  media }\OtherTok{\textless{}{-}} \FunctionTok{mean}\NormalTok{(data)}
\NormalTok{  ic }\OtherTok{\textless{}{-}}\NormalTok{ media}\SpecialCharTok{+}\FunctionTok{c}\NormalTok{(}\SpecialCharTok{{-}}\DecValTok{1}\NormalTok{,}\DecValTok{1}\NormalTok{)}\SpecialCharTok{*}\FloatTok{1.96}\SpecialCharTok{*}\NormalTok{(}\DecValTok{1}\SpecialCharTok{/}\FunctionTok{sqrt}\NormalTok{(}\DecValTok{100}\NormalTok{))}
\NormalTok{  ic[}\DecValTok{1}\NormalTok{] }\SpecialCharTok{\textless{}} \DecValTok{0} \SpecialCharTok{\&}\NormalTok{ ic[}\DecValTok{2}\NormalTok{] }\SpecialCharTok{\textgreater{}} \DecValTok{0}
\NormalTok{\})}
\FunctionTok{mean}\NormalTok{(result)}
\end{Highlighting}
\end{Shaded}

\begin{verbatim}
## [1] 0.955
\end{verbatim}

\hypertarget{utiliza-la-funciuxf3n-t.test-de-r-para-valorar-si-encuentras-diferencias-en-las-medias-de-las-poblaciones-de-las-que-provienen-las-siguientes-2-muestras-set.seed1x-rnorm10-e-y-rnorm101.-eleva-el-tamauxf1o-muestral-de-ambas-muestras-a-20-y-30-para-valorar-como-cambian-tus-conclusiones.}{%
\subsection{4 Utiliza la función t.test de R para valorar si encuentras
diferencias en las medias de las poblaciones de las que provienen las
siguientes 2 muestras: set.seed(1);x\textless-rnorm(10) e
y\textless-rnorm(10,1). Eleva el tamaño muestral de ambas muestras a 20
y 30 para valorar como cambian tus
conclusiones.}\label{utiliza-la-funciuxf3n-t.test-de-r-para-valorar-si-encuentras-diferencias-en-las-medias-de-las-poblaciones-de-las-que-provienen-las-siguientes-2-muestras-set.seed1x-rnorm10-e-y-rnorm101.-eleva-el-tamauxf1o-muestral-de-ambas-muestras-a-20-y-30-para-valorar-como-cambian-tus-conclusiones.}}

\begin{Shaded}
\begin{Highlighting}[]
\NormalTok{ns }\OtherTok{\textless{}{-}} \FunctionTok{c}\NormalTok{(}\DecValTok{10}\NormalTok{, }\DecValTok{20}\NormalTok{, }\DecValTok{30}\NormalTok{)}
\NormalTok{result }\OtherTok{\textless{}{-}} \FunctionTok{sapply}\NormalTok{(ns, }\ControlFlowTok{function}\NormalTok{(x)\{}
  \FunctionTok{set.seed}\NormalTok{(}\DecValTok{1}\NormalTok{)}
\NormalTok{  xx }\OtherTok{\textless{}{-}} \FunctionTok{rnorm}\NormalTok{(x)}
\NormalTok{  yy }\OtherTok{\textless{}{-}} \FunctionTok{rnorm}\NormalTok{(x, }\DecValTok{1}\NormalTok{)}
  \FunctionTok{t.test}\NormalTok{(xx, yy)}\SpecialCharTok{$}\NormalTok{p.value}
\NormalTok{  \})}
\NormalTok{result}
\end{Highlighting}
\end{Shaded}

\begin{verbatim}
## [1] 0.0165778586 0.0071229035 0.0000159436
\end{verbatim}

\end{document}
